\subsection{Audio module implementation}

The structure of the program deviates a bit from the three layers described in
\cite{Lect2} on slide 9, there is no application interface layer - as seen in
figure:~\ref{fig:overview}

\stdfig{0.8}{audiomodule/overview}{Block diagram of overall program
structure}{fig:overview}

The main reasons for this is that the application had to be ported to
\emph{Windows} for testing.
since it was not possible to run the application at real-time on a PC, the
program ran for $x$ samples and this was then saved to a wave-file.
Core hardware features on the \emph{Stellaris Launchpad} are simply missing on a
computer, but making the layer in this way paves the way for easy portability (only
hardInterface.c and to a limited degree hardInterface.h has to be changed).

This layer might more accurately be called a glue layer, as it does not
interface directly with the hardware:
\begin{itemize}
  \item On the \emph{Stellaris Launchpad} the \emph{HIL} interfaces with a
  driver library written by \emph{Texas Instruments} - valid for all
  \emph{MCU's} from the same family. These functions then interface with the
  hardware.
  \item On \emph{Windows} the \emph{HIL} interfaces with a wave-file library
  \cite{writewav}, which then in turn uses \emph{stdio.h}, that interfaces with
  the operating system, interfacing with the hard-drive driver \ldots and so
  on.
\end{itemize}

The rest of the document will primarily focus on implementation on the
\emph{Stellaris Launchpad}.

\subsubsection{Hardware Interface Layer (\emph{HIL})}
The front end of this layer is very simple, four functions are presented:
\begin{itemize}
  \item void initiateHardware(): This function is used to initiate the hardware.
  Sets up PLL, timers, GPIO and 44kHz interrupt on Stellaris.
  \item void busyWaiting( FIFOBuffer* midiBuffer ): This functions waits until
  the next output should be calculated, while moving data to the midiBuffer
  FIFO.
  \item void writeBuffer( int32\_t signal ): This functions writes the current
  output to the output buffer.
  \item void loopCondition(): Returns 1 if the program should continue execution
  and 0 if it should terminate. On the \emph{Stellaris Launchpad} this is
  replaced by a \emph{\#define}, replacing the function with a constant 0.
\end{itemize}

To prove consistency in the output stream, the signal is outputted at the
beginning of the interrupt - then the process returns to the main loop where the
next output is calculated - the processor then waits for the next interrupt.

\subsubsecton
