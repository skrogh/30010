\subsection{Audio module implementation}

The structure of the program deviates a bit from the three layers described in
\cite{Lect2} on slide 9, there is no application interface layer - as seen in
figure:~\ref{fig:overview}

\stdfig{0.8}{audiomodule/overview}{Block diagram of overall program
structure}{fig:overview}

The main reasons for this is that the application had to be ported to
\emph{Windows} for testing.
since it was not possible to run the application at real-time on a PC, the
program ran for $x$ samples and this was then saved to a wave-file.
Core hardware features on the \emph{Stellaris Launchpad} are simply missing on a
computer, but making the layer in this way paves the way for easy portability (only
hardInterface.c and to a limited degree hardInterface.h has to be changed).

This layer might more accurately be called a glue layer, as it does not
interface directly with the hardware:
\begin{itemize}
  \item On the \emph{Stellaris Launchpad} the \emph{HIL} interfaces with a
  driver library written by \emph{Texas Instruments} - valid for all
  \emph{MCU's} from the same family. These functions then interface with the
  hardware.
  \item On \emph{Windows} the \emph{HIL} interfaces with a wave-file library
  \cite{writewav}, which then in turn uses \emph{stdio.h}, that interfaces with
  the operating system, interfacing with the hard-drive driver \ldots and so
  on.
\end{itemize}

The rest of the document will primarily focus on implementation on the
\emph{Stellaris Launchpad}.

\subsubsection{FIFO buffer}
This is a general library for creating \emph{FIFO} buffers in software.
The buffer is created as a structure and only the structure itself and the
function to initiate it is included in the header-file - all functions performed
by the buffer is kept as function pointers in the structure, and can thus not be
called without a reference from the structure. This somewhat simulates object
oriented programming and makes the code more manageable.

\begin{lstlisting}[caption={Relevant part of FIFO.h}, language=C,
label=lst:buffer] struct tag_FIFOBuffer { //FIFO buffer
	uint32_t* data; // data array
	uint32_t length; // array size
	uint32_t pointer; // address (in array) of next element
	uint32_t elements; // number of  elements in fifo
	/*
	 * push @data into the buffer, returns 1 on success, 0 on full buffer
	 */
	uint32_t (*push)( FIFOBuffer* fifoBuffer, uint32_t data );
	/*
	 * Pops a data element from the buffer into the address specified by the pointer @data.
	 * If there is no data to pop, 0 is returned and @data is not changed, else 1 is returned.
	 */
	uint32_t (*pop)( FIFOBuffer* fifoBuffer, uint32_t* data );
	/*
	 * Peaks at a data element from the buffer and puts it into the address specified by the pointer @data,
	 * without removing it from the buffer.
	 * @depth specifies how far to look into the buffer.
	 * If there is no data to pop, 0 is returned and @data is not changed, else 1 is returned.
	 */
	uint32_t (*peak)( FIFOBuffer* fifoBuffer, uint32_t* data, uint32_t depth ); // peak at the element depth into the stack (0 = top, 1 = just under top) - returns 1 if that element exists.
};
\end{lstlisting}

The  relevant part of the \emph{FIFO} header file is seen above in
listing:~\ref{lst:buffer}. It can be seen that the \emph{peak} and \emph{pop}
does not return the value of the data element directly, but returns a value
indicating success or failure - while the data element is returned by changing
the value of the variable pointed to by \emph{data}. This ensures the
capability of error handling (a bit like throwing an exception in \emph{Java}).

\stdfig{0.8}{audiomodule/FIFO}{
Graphical description of the FIFO buffer. The green arrow represents the reading
head (\emph{pointer}).\\
In \emph{A} the buffer is initialized with a length of 4.\\
In \emph{B} 3 elements $[1,~2,~3]$ are pushed into the buffer.\\
In \emph{C} the 3 elements are popped out again in order $[1,~2,~3]$.\\
In \emph{D} 4 more elements are pushed in.\\
In \emph{E} 3 elements has been popped out, notice how the reading head loops
around the edge of the buffer.
 }{fig:fifo}

The \emph{FIFO} buffer is initialized with the function \emph{createFIFOBuffer}.
The function is passed a pointer to the buffer to initiate, a pointer to an
array for storing the data and a value for the length of the buffer.

As there is no need to add or remove buffers dynamically the \emph{malloc}
function is not used for this. 

\subsubsection{Hardware Interface Layer (\emph{HIL})}
The front end of this layer is very simple, four functions are presented:
\begin{itemize}
  \item void initiateHardware(): This function is used to initiate the hardware.
  Sets up PLL, timers, GPIO and 44kHz interrupt on Stellaris.
  \item void busyWaiting( FIFOBuffer* midiBuffer ): This functions waits until
  the next output should be calculated, while moving data to the midiBuffer
  FIFO.
  \item void writeBuffer( int32\_t signal ): This functions writes the current
  output to the output buffer.
  \item void loopCondition(): Returns 1 if the program should continue execution
  and 0 if it should terminate. On the \emph{Stellaris Launchpad} this is
  replaced by a \emph{\#define}, replacing the function with a constant 0.
\end{itemize}

To prove consistency in the output stream, the signal is outputted at the
beginning of the interrupt - then the process returns to the main loop where the
next output is calculated - the processor then waits for the next interrupt.

\paragraph{Timing LED}
We had some problems with the calculations not always having time to calculate
within the given time frame - so an LED was put to light up green, while the
processor is waiting for the next interrupt. If the LED light is very dim - or
not lighting at all - it indicates that too much code has to execute and either
optimizations or reductions has to be made to the code.


\subsubsection{implementation of the synth modules}
The synthesizer modules are built as encapsulated structures, like the FIFO
buffer.

This allows for easy, fast and structured construction of instruments.

All the modules are kept in a folder named \emph{synth\_lib}

\paragraph{VCO}
The \emph{VCO} works by incrementing a counter by a value $value =
44000 \times \frac{frequency in Hz}{2^{32}}$ for each update. The natural
overflow of the counter becomes the saw wave output. The frequency is found in a
pre-generated look up table (array) with the \emph{freq} as the index. 

\paragraph{LFO}
The \emph{LFO} is functionally similar to the \emph{VCO}, but the speed is set
$2^7$ times slower - thus going down to about $0.1Hz$. On overflow of the
counter the \emph{tick} output is set high.

\paragraph{Sequencer}
The sequencer simply counts its way through an array, looping back to the start,
when it reaches the end.

\paragraph{LowPass}
The algorithm for the IIR low-pass filter is taken from the description given
in \cite{Lowpass}. The implantation thus consists of transferring this to
written code. The code itself can be found in the appendix.

Since fixed-point arthritics are used, multiplication is performed with 64 bit
variables to avoid overflow

\paragraph{SVF}
The algorithm for the state-variable filter is taken from \cite{SVF}.

As the gain of the filter can exceed 1 at the cutoff frequency, it is important
to reduce the volume of the signal going into the filter.

\paragraph{AHDSR}
The \emph{AHDSR} update function is implemented with a \emph{switch}
structure, with one case for each of the 5 phases.

\paragraph{WaveShaper}
The wave-shaper is a bit more technical.
It has the capability to pic different tables, according to the input
frequency, thus generating bandwidth limited waves - however this proved to be
both very processor heavy (quadratic interpolation was planned, but dropped
because of lacing processor power) and the wave tables had to be extremely
large, to not have audible ``jumps'' when switching.
Thus the capability of changing wave shape according to frequency remains, but is
unused.

\begin{itemize}
  \item The wave-shaper first determines the frequency, by comparing the last
  input value with the current.
  \item Then the correct point in the wave-table is found - this is being done
  by converting the signal to fixed-point with precision varying according to
  the wave-table length.
  \item Everything right of the zero is truncated (this data
  would have been used for interpolation) and the correct data-point in the
  table is picked.
\end{itemize}

\paragraph{MidiInterpreter}
As this document will not deal with the definitions of \emph{MIDI} a short
explanation of the protocol can be found here \cite{MIDI}.

In short \emph{MIDI} consists of 8 bit messages, first comes a control byte,
always starting with 1 as the MSB. Then 3 bits to describe the message type and
finally 3 bits to indicate channel ($0xF?$ indicates a system exclusive message
and the channel $?$ is used as an extra parameter). Then follows one or more
value messages. The MSB is always 0 for those.

We have chosen only to implement note on channel 1, control byte: $0x80$,
followed by the note. This will play the different game sounds
Start:

\paragraph{OneShotter}
The \emph{OmeShotter} can play back wave-files at different speeds. A counter
counts up and on overflow the playback head moves, this playback head is used as
index in an array storing the wave-file. The frame being pointed to is then
output.



