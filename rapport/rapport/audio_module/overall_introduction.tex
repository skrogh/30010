\subsection{Audio module introduction}
The audio module will be run as an external application on a \emph{Texas
Instruments} \emph{Stellaris Launchpad}, communicating with the \emph{eZ8} via
\emph{SPI}.

The audio module performs the following:
\begin{itemize}
  \item Play and stop one song: \emph{Popcorn}
  \item The song has 3 voices
  \begin{itemize}
    \item Lead
    \item Bass
    \item Percussion
  \end{itemize}
  \item The module connects to the \emph{eZ8} with SPI and receives data as
  \emph{MIDI} messages.
  \item The module can play game specific sounds (bleeps when the ball hits
  anything and similar)
\end{itemize}

Since this section assumes some known knowledge about synthesizers, a small
introduction to subtractive synthesis is given in the next section.

\subsubsection{An introduction to subtractive synthesis}\label{synthxp}
Subtractive synthesis is one of the very popular methods of creating musical
notes. A very basic synthesizer is described in figure:~\ref{fig:basicsynth}

\stdfig{0.8}{audiomodule/basicsynth}{Block diagram of a basic
subtractive synthesizer, with examples of signal waveforms. A
sequencer is used as the source of notes. Note that the timing
of the sequencer signal examples is not one scale with the
rest of the signal examples.}{fig:basicsynth}

A steady stream of pulses enters the sequence and makes it advance through it's
pattern, here with a pattern length of 5. The sequencer output is used as the
input of the oscillator to control the frequency of it - this could just as well
be from a keyboard - The oscillator generates a harmonic rich output that is fed
into a resonant filter (2 pole low-pass here) to \emph{subtract} harmonics; then
the filtered signal is fed through an amplifier.

The cutoff frequency of the filter is controlled by the \emph{ADSR}, an acronym
for \emph{Attack Decay Sustain Release} this works as following:
\begin{itemize}
  \item Attack: After a trigger the output rises from 0
to max at a specified rate.
  \item Decay: The signal then falls to the sustain level at a specified rate.
  \item Sustain: As long ad the gate is high, the output is helt at the sustain
  level.
  \item Release: Once the gate is low, the output falls back to 0 at a specified
  rate.
\end{itemize}

The amplitude of the amplifier is controlled by an \emph{AHDSR}, the same as the
\emph{ADSR}, except a short delay, where the output is held at max - the hold
phase - is put in between the attack and decay stages.
 