\subsection{Testing the audio module}
As mentioned, the application was made so it could run smoothly on both the
\emph{Stellaris Launchpad} an on \emph{Windows}. Having the possibility to run
the application and saving a \emph{.WAV} file with the output was a huge
benefit, as the debugging tool only provided momentary debugging, but not the
possibility to view the output as a whole.
The testing could thus be performed as a mix of both listening and visual
inspection. A picture from the output wave-form can be seen in
figure~\ref{fig:wve}

\stdfig{0.8}{audiomodule/wve}{An excerpt of the output file, generated by
playing \emph{Popcorn}, at the beginning of the dark-gray area we see a
percussive sound (burst of noise) then we see the bass synth square-wave with
an applied resonant low-pass filter. The small ``jitters'' are the lead
voice}{fig:wve}

The only way to thoroughly test this system would be to make small test
applications, testing each module - this however would be a very slow process
and would take up a lot of space in the report. It was somewhat done
mid-project, when something didn't work as expected the application would be
rewritten to only feature the troubling module to isolate and remove the error
- but this was not documented in due time.

\subsection{Audio module results}
Without connection to the game, the audio module cannot do much - as it lacks
the message to start the music - and to play the sound effects.
This can - however - be bypassed, when using the debugging tool; we can change
the vale of registers to ``cheat'' the audio module into thinking, it has gotten
the required messages.

The Popcorn song can be herd in the included files: \emph{PopcornStellaris.WAV}
and \emph{PopcornWindows.WAV}.
They can also be found on the following webpages:\\
Sounds from the \emph{Stellaris Launchpad}:
\url{https://soundcloud.com/skrogh/popcorn-stellaris}\\
Sounds from \emph{Windows}: \url{https://soundcloud.com/skrogh/popcorn-windows}


