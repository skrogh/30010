\subsubsection{implementation of the synth modules}
The synthesizer modules are built as encapsulated structures, like the FIFO
buffer.

This allows for easy, fast and structured construction of instruments.

All the modules are kept in a folder named \emph{synth\_lib}

\paragraph{VCO}
The \emph{VCO} works by incrementing a counter by a value $value =
44000 \times \frac{frequency in Hz}{2^{32}}$ for each update. The natural
overflow of the counter becomes the saw wave output. The frequency is found in a
pre-generated look up table (array) with the \emph{freq} as the index. 

\paragraph{LFO}
The \emph{LFO} is functionally similar to the \emph{VCO}, but the speed is set
$2^7$ times slower - thus going down to about $0.1Hz$. On overflow of the
counter the \emph{tick} output is set high.

\paragraph{Sequencer}
The sequencer simply counts its way through an array, looping back to the start,
when it reaches the end.

\paragraph{LowPass}
The algorithm for the IIR low-pass filter is taken from the description given
in \cite{Lowpass}. The implantation thus consists of transferring this to
written code. The code itself can be found in the appendix.

Since fixed-point arthritics are used, multiplication is performed with 64 bit
variables to avoid overflow

\paragraph{SVF}
The algorithm for the state-variable filter is taken from \cite{SVF}.

As the gain of the filter can exceed 1 at the cutoff frequency, it is important
to reduce the volume of the signal going into the filter.

\paragraph{AHDSR}
The \emph{AHDSR} update function is implemented with a \emph{switch}
structure, with one case for each of the 5 phases.

\paragraph{WaveShaper}
The wave-shaper is a bit more technical.
It has the capability to pic different tables, according to the input
frequency, thus generating bandwidth limited waves - however this proved to be
both very processor heavy (quadratic interpolation was planned, but dropped
because of lacing processor power) and the wave tables had to be extremely
large, to not have audible ``jumps'' when switching.
Thus the capability of changing wave shape according to frequency remains, but is
unused.

\begin{itemize}
  \item The wave-shaper first determines the frequency, by comparing the last
  input value with the current.
  \item Then the correct point in the wave-table is found - this is being done
  by converting the signal to fixed-point with precision varying according to
  the wave-table length.
  \item Everything right of the zero is truncated (this data
  would have been used for interpolation) and the correct data-point in the
  table is picked.
\end{itemize}

\paragraph{MidiInterpreter}
As this document will not deal with the definitions of \emph{MIDI} a short
explanation of the protocol can be found here \cite{MIDI}.

In short \emph{MIDI} consists of 8 bit messages, first comes a control byte,
always starting with 1 as the MSB. Then 3 bits to describe the message type and
finally 3 bits to indicate channel ($0xF?$ indicates a system exclusive message
and the channel $?$ is used as an extra parameter). Then follows one or more
value messages. The MSB is always 0 for those.

We have chosen only to implement note on channel 1, control byte: $0x80$,
followed by the note. This will play the different game sounds
Start:
