\subsection{Audio module design}

As described under the analysis, the audio module has to be capable of both
generating sounds and playing predefined wave-files. The processor is chosen to
run in real-time with a frequency of 44kHz, this is done to reduce aliasing
artifacts - in theory this is enough to allow an output spanning the full
audible range, given sufficient filtering and taking care not to generating any
tones above the nyquist-frequency; we did however not have the time to dive
fully into this theory and thus some aliasing will occur.

The design of the audio module will be done like a digital model of a modular
synthesizer, taking inspiration from eg. the free-to-try software
\emph{SynthEdit0} in which different block (oscillators, filters, transient
generators, wave-file players etc.) can be connected to form whole synthesizers.
See figure:~\ref{fig:synthedit}.

This approach allows for both easy programming of the modules, as this becomes
an highly manageable problem; easy programming of the synthesizers, this is just
connecting the blocks; and provides an easily expandable and reusable library.

Communication is done over SPI, using MIDI (\emph{Musical Instrument Digital
Interface}) commands. This allows for a MIDI interpreter module, working in
similar ways as all other modules.

\subsubsection{Overview}
We decided that the background music should be the 80'es classic
\emph{Popcorn}.
For this we need three voices, lead, bass and percussion.
The score is stored in 6 arrays (notes and gate/trigger for each voice) and a
sequencer goes through this array, giving notes and triggers to the synthesisers
and wave-player. This is illustrated in figure:~\ref{fig:audioflow} 

\stdfig{0.8}{audiomodule/audioflow}{Block diagram of the audio
module}{fig:audioflow}




\stdfig{0.8}{audiomodule/synthedit.jpg}{The design of the synthesizers took
inspiration from the block approach as seen in eg. the free program
\emph{Synth Edit} }{fig:synthedit}
%TODO add reference to http://www.soundonsound.com/sos/sep09/articles/vstdiy.htm
%TODO add reference to
% http://www.synthedit.com/